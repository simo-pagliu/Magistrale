\chapter{Pulse Mode}

\section{Pulse Mode Operation}
In pulse mode operation, the reactor is driven into a short, high-intensity power pulse. This mode is characterized by:
\begin{itemize}
    \item Short duration
    \item High intensity
\end{itemize}
The observed parameters in pulse mode include:
\begin{enumerate}
    \item Power (Intensity)
    \item Time to reach peak power (to compare pulse duration)
    \item Full Width at Half Maximum (FWHM) of the pulse
\end{enumerate}

\begin{figure}[H]
    \centering
    \includegraphics[width=\textwidth]{placeholder.png}
    \caption{Illustration of pulse mode operation with parameters like peak power and FWHM.}
    \label{fig:pulse_mode}
\end{figure}

\section{Functions of Pulse Mode}
Pulse mode can be used for various purposes, including:
\begin{itemize}
    \item Activation analysis, including detector or semiconductor analysis
    \item Safety tests, such as reactivity insertion accidents
    \item Estimation of parameters related to reactor kinetics
\end{itemize}

\section{Physical Background}
In the pulse mode, the following sequence occurs:
\begin{align*}
    \rho \uparrow \to P \uparrow \to E(T) \to T \uparrow \to \alpha_T \to \rho \downarrow \to \text{Feedback stabilizes the system.}
\end{align*}
To translate energy into temperature, we need to know the heat capacity, $C_p$. The other important parameter is the temperature reactivity coefficient, $\alpha_f$.

\section{Experimental Evidence of Ljubljana Pulses}
Videos of different reactivity insertions in the Ljubljana TRIGA reactor show that:
\begin{itemize}
    \item Higher intensity in Cherenkov radiation correlates with larger $\Delta \rho$.
    \item Shorter pulses are achieved with larger $\Delta \rho$.
    \item The reactor is always scrammed after each pulse for safety, even if it is self-limited.
\end{itemize}
The Nordheim-Fuchs model is employed to understand the self-limited excursions during these pulses. The model incorporates several key assumptions:
\begin{enumerate}
    \item Transients are modeled using point kinetics.
    \item Delayed neutrons are neglected due to the short timescale.
    \item The reactivity insertion, $\Delta \rho$, is treated as a step function.
    \item Adiabatic heat transfer is assumed for the energy balance, focusing on the heat transfer to the coolant over a characteristic timescale.
\end{enumerate}
The adiabatic assumption is valid only from a global perspective, as local variations in fission rates across individual fuel elements can lead to higher localized heat transfer coefficients (HTCs). This highlights the importance of understanding local phenomena in addition to the overall system behavior.

\section{Nordheim-Fuchs Model}
The Nordheim-Fuchs model is used to understand self-limited excursions over a short timespan. It relies on several assumptions:
\begin{enumerate}
    \item Transient modeled with point kinetics
    \item Delayed neutrons are neglected
    \item $\Delta \rho$ is considered a step function
    \item Adiabatic model for energy balance
\end{enumerate}

\begin{figure}[H]
    \centering
    \includegraphics[width=\textwidth]{placeholder.png}
    \caption{Power pulse prediction by Nordheim-Fuchs model showing peak power and pulse duration.}
    \label{fig:nordheim_fuchs}
\end{figure}

\subsection{Derivation of the Nordheim-Fuchs Model}
The derivation of the Nordheim-Fuchs model begins with several simplifying assumptions:
\begin{enumerate}
    \item Point kinetics approximation is used, ignoring spatial effects.
    \item Delayed neutrons are neglected ($C_i = 0$) due to the short timescale of the pulse.
    \item No external neutron source is present ($S = 0$).
    \item Reactivity insertion, $\rho(t)$, is treated as a step function: $\rho(t) = \rho_0 H(t)$.
    \item Heat transfer is modeled adiabatically, assuming no heat loss during the transient.
\end{enumerate}
The rate of change of reactivity is given by:
\begin{align*}
    \frac{dP}{dt} = \frac{\rho - \beta}{\Lambda}P,
\end{align*}
where $P$ is the power, $\beta$ is the delayed neutron fraction, and $\Lambda$ is the prompt neutron lifetime. 

For the temperature feedback:
\begin{align*}
    \rho = \rho_0 - \alpha(T_f - T_{f,0}),
\end{align*}
where $\alpha$ is the temperature reactivity coefficient and $T_f$ is the fuel temperature. Using the adiabatic heat transfer model:
\begin{align*}
    \frac{dT_f}{dt} = \frac{P}{m_f c_{p,f}},
\end{align*}
where $m_f$ is the fuel mass and $c_{p,f}$ is the specific heat capacity. Combining these equations leads to a coupled differential system that can be solved for $P(t)$ and $T_f(t)$.

After integration, the peak power and the Full Width at Half Maximum (FWHM) of the pulse can be derived:
\begin{align*}
    P_{\text{max}} &= \frac{m_f c_{p,f}}{\alpha}\left(\rho_0^2 - \beta^2\right), \\
    \text{FWHM} &= 3.52\sqrt{\frac{\Lambda}{\alpha m_f c_{p,f}}}.
\end{align*}
These expressions highlight the dependence of pulse characteristics on reactor parameters such as $\rho_0$, $\alpha$, and $m_f$.

\section{Power Pulse Parameters}
The key parameters of a power pulse are:
\begin{itemize}
    \item $\Delta \rho$: Known insertion value
    \item $T_f$: Fuel temperature from instrumentation
    \item $E_{\text{released}}$: Power integral for total energy release
\end{itemize}

\section{Why TRIGA Can Operate in Pulse Mode}
The TRIGA reactor can safely operate in pulse mode due to:
\begin{itemize}
    \item Large prompt and negative temperature coefficient, $\alpha_f = -8$ to $-10$~pcm/K
    \item High heat capacity of the fuel-moderator combination
\end{itemize}
Note: In the Ljubljana TRIGA reactor, fewer control rods and elements with higher uranium content allow a unique pulse operation configuration.

\section{Physical Phenomenon of the Pulse}
The physical phenomenon during a pulse involves:
\begin{enumerate}
    \item Large $\rho$ insertion ($> 2\$$)
    \item Prompt supercriticality reached
    \item Fuel temperature increases, causing $\rho \downarrow$ until stability
\end{enumerate}

\begin{figure}[h]
    \centering
    \includegraphics[width=\textwidth]{placeholder.png}
    \caption{Graph showing power pulse parameters and energy release.}
    \label{fig:power_pulse}
\end{figure}

\section{Examples and Limitations}
Examples of the results from the Nordheim-Fuchs model indicate:
\begin{itemize}
    \item Increasing $\rho_0$ leads to higher peak power, reduced FWHM, increased energy in fuel, and larger $\Delta T_f$.
\end{itemize}

Limitations of this model include:
\begin{itemize}
    \item For slow pulses, delayed neutrons are no longer negligible, and their impact must be accounted for.
    \item The heat capacity, $C_p$, actually depends on the temperature, introducing non-linearities that the model does not capture.
    \item The model does not account for ramped reactivity insertions, assuming a step-like reactivity input instead.
    \item The point kinetics assumption ignores spatial effects, which may become significant in certain configurations.
    \item Experimental validation shows deviations in scenarios where local thermal feedback or hydrodynamic effects dominate.
\end{itemize}

\section{Experimental Procedure}
The following steps outline the experimental procedure for conducting pulse mode experiments:
\begin{enumerate}
    \item Establish critical conditions with the transient rod inserted, then fully withdraw the transient rod to initiate the pulse.
    \item Maintain the reactor at low power (approximately 1000 W) to avoid shadowing effects during the step insertion.
    \item Fire the pulse by extracting the transient rod using the fast pneumatic system.
    \item Scram the reactor approximately 15 seconds after the pulse to ensure safety.
    \item Wait for the reactor to cool down before proceeding with further operations.
\end{enumerate}
