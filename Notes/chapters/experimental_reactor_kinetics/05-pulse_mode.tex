\section{Pulse Mode Operation}

In pulse mode operation, the reactor is driven into a short, high-intensity power pulse. This mode is characterized by:
\begin{itemize}
    \item Short duration
    \item High intensity
\end{itemize}

The observed parameters in pulse mode include:
\begin{enumerate}
    \item Power (Intensity)
    \item Time to reach peak power (to compare pulse duration)
    \item Full Width at Half Maximum (FWHM) of the pulse
\end{enumerate}

\begin{figure}[h]
    \centering
        \includegraphics[width=0.75\linewidth]{placeholder.png}
    \caption{Graph illustrating pulse mode operation with parameters like peak power and FWHM.}
\end{figure}

\section{Functions of Pulse Mode}

Pulse mode can be used for various purposes, including:
\begin{itemize}
    \item Activation analysis, including detector or semiconductor analysis
    \item Safety tests, such as reactivity insertion accidents
    \item Estimation of parameters related to reactor kinetics
\end{itemize}

\section{Physical Background}

In the pulse mode, the following sequence occurs:
\[
\rho \uparrow \rightarrow P \uparrow \rightarrow E(T) \rightarrow T \uparrow \rightarrow \alpha_T \rightarrow \rho \downarrow \rightarrow \text{Feedback stabilizes the system.}
\]

To translate energy into temperature, we need to know the heat capacity, $C_p$. The other important parameter is the temperature reactivity coefficient, $\alpha_f$.

\section{Experimental Evidence of Ljubljana Pulses}

Videos of different reactivity insertions in the Ljubljana TRIGA reactor show that:
\begin{itemize}
    \item Higher intensity in Cherenkov radiation correlates with larger $\Delta \rho$.
    \item Shorter pulses are achieved with larger $\Delta \rho$.
    \item The reactor is always scrammed after each pulse for safety, even if it’s self-limited.
\end{itemize}

\section{Nordheim-Fuchs Model}

The Nordheim-Fuchs model is used to understand self-limited excursions over a short timespan. It relies on several assumptions:
\begin{enumerate}
    \item Transient modeled with point kinetics
    \item Delayed neutrons are neglected
    \item $\Delta \rho$ is considered a step function
    \item Adiabatic model for energy balance
\end{enumerate}

\begin{equation}
    % MISSING EQUATION
\end{equation}

\begin{figure}[h]
    \centering
        \includegraphics[width=0.75\linewidth]{placeholder.png}
    \caption{Power pulse prediction by Nordheim-Fuchs model showing peak power and pulse duration.}
\end{figure}

\section{Power Pulse Parameters}

The key parameters of a power pulse are:
\begin{itemize}
    \item $\Delta \rho$: Known insertion value
    \item $T_f$: Fuel temperature from instrumentation
    \item $E_{\text{released}}$: Power integral for total energy release
\end{itemize}

\section{Why TRIGA Can Operate in Pulse Mode}

The TRIGA reactor can safely operate in pulse mode due to:
\begin{itemize}
    \item Large prompt and negative temperature coefficient, $\alpha_f = -8$ to $-10$ pcm/K
    \item High heat capacity of the fuel-moderator combination
\end{itemize}

\begin{tcolorbox}[colback=white, colframe=cherenkovblue, boxrule=0.5mm, sharp corners]
\textbf{Note:} In the Ljubljana TRIGA reactor, fewer control rods and elements with higher uranium content allow a unique pulse operation configuration.
\end{tcolorbox}

\begin{figure}[h]
    \centering
        \includegraphics[width=0.75\linewidth]{placeholder.png}
    \caption{Graph showing power pulse parameters and energy release.}
\end{figure}

\section{Physical Phenomenon of the Pulse}

The physical phenomenon during a pulse involves:
\begin{enumerate}
    \item Large $\rho$ insertion ($>2$)
    \item Prompt supercriticality reached
    \item Fuel temperature increases, causing $\rho \downarrow$ until stability
\end{enumerate}

\begin{figure}[h]
    \centering
        \includegraphics[width=0.75\linewidth]{placeholder.png}
    \caption{Illustration of power and energy evolution in pulse mode.}
\end{figure}

\section{Power Pulse Parameters}

The parameters for a power pulse are typically:
\begin{itemize}
    \item $\Delta P$: Known insertion level
    \item $T_f$: Measured from instrumentation
    \item $E_{\text{released}}$: Calculated from power integral
\end{itemize}

\section{Examples and Limitations}

Examples of the results from the Nordheim-Fuchs model indicate:
\begin{itemize}
    \item Increasing $\rho_0$ leads to higher peak power, reduced FWHM, increased energy in fuel, and larger $\Delta T_f$.
\end{itemize}

Limitations of this model include:
\begin{itemize}
    \item Valid only for short pulses (delay of neutrons is negligible)
    \item Incomplete modeling of the temperature dependency
\end{itemize}

\begin{tcolorbox}[colback=white, colframe=cherenkovblue, boxrule=0.5mm, sharp corners]
\textbf{Note:} The Nordheim-Fuchs model requires validation and verification against experimental data for accurate results.
\end{tcolorbox}
