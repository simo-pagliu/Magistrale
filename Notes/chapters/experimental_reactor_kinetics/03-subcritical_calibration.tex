\section{Subcritical Control Rod Calibration}

\subsection{Introduction}
The subcritical method provides an alternative approach to control rod calibration, particularly useful for cases where safety is a concern. 
This method allows for calibration while keeping the reactor in a subcritical state by using an external neutron source.

\subsection{Theory - TBC}
\paragraph{Subcritical Multiplication and Reactor Period}
In subcritical conditions, the neutron population depends on both the external source and the reactor’s multiplication factor $K_{eff}$. This dependency is described by the subcritical multiplication factor $M$:
\begin{equation}
    M = \frac{1}{1 - K_{eff}}
\end{equation}
With an external source, the neutron population evolves over generations until it reaches a steady state. 
This evolution can be described by a geometric series with $K_{eff}$ as the ratio, where the number of generations required to reach steady state is:
\begin{equation}
    N_{\text{steady}} \approx \frac{4}{\ln K}
\end{equation}
Which means that the evolution is slower the closer we get to the steady state condition.

\paragraph{Point Kinetics Equations}
To understand the reactor's response over time, we use the point kinetics equations adapted for the subcritical state:
\begin{equation}
    \frac{dn}{dt} = K_{eff} \left( \frac{K_{eff} - 1}{K_{eff}} + \rho - \beta \right) \frac{n}{\Lambda} + \sum \lambda_i C_i + q
\end{equation}
\begin{equation}
    \frac{dC_i}{dt} = K_{eff} \frac{\beta_i}{\Lambda} n - \lambda_i C_i
\end{equation}
where $\rho = \frac{\Lambda}{K}$ and $q$ is the source term. The reactor period $T$ is then given by:
\begin{equation}
    T = \frac{1}{\lambda_1}
\end{equation}

\subsection{Experimental Procedure}
The following steps outline the subcritical calibration procedure:
\begin{enumerate}
    \item Bring the reactor in subcritical condition with all rods inserted.
    \item Measure the neutron rate $\dot{R}$, we used a fission chamber, in pulse mode.
    \item Insert the source and measure $\dot{R}$ again, we used $Ra\text{-}Be$,.
    \item Extract the control rods of which we know its reactivity worth and measure $\dot{R}$ once it reached steady state.
    \item Compute the subcritical multiplication factor of the CR from known reactivity worth: $\Delta \rho = \rho_{out} - 0 = \frac{K - 1}{K} \rightarrow M = \frac{1}{1 - K}$.
    \item Compute the calibration parameter $\alpha \Phi_s$ from this measurment: $\frac{\dot{R}}{M} = \alpha \Phi_s$.
    \item Reinsert the control rod.
    \item Extract another control rod and measure $\dot{R}$.
    \item Compare measured $\dot{R}$ to the calibration parameter: $M = \alpha \Phi_s \dot{R}$
    \item Repeat the last two steps for all control rods.
\end{enumerate}
For better statistics take multiple measurements for each control rod (or long measurements).

\begin{tcolorbox}[boxstyle2]
    \textbf{Note on statistics}:
    In poisson distributed data like radiation counts taking 20 short measurments each of time $t$ will give the same statistical error
    as taking 1 longer measurement of time $T = 20t$.
\end{tcolorbox}


\subsection{Pros and Cons of the Subcritical Method}
    \textbf{Advantages}:
    \begin{itemize}
        \item Safe and keeps the reactor subcritical.
        \item Allows calibration of the shim, unlike other methods.
    \end{itemize}
    \textbf{Disadvantages}:
    \begin{itemize}
        \item Relys on the knowledge of one control rod's reactivity worth.
        \item Lower accuracy due to reliance on source and instrumentation.
        \item Only measures the total worth of the control rod, unable to get the full integral curve with reasonable measuring time (if we do a small step the variation in count rate change can easly hide in the noise and error).
        \item Frequent recalibration required due to burnup of fission chamber.
    \end{itemize}