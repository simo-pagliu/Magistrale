\section{Critical Calibration Method}

\subsection{Theory}
The calibration of control rods in a reactor is crucial for understanding their reactivity worth. This can be accomplished using the in-hour and point kinetics equations to relate the time required to increase power by a factor to the reactivity.

\paragraph{Reactivity and Reactor Period}
The relationship between reactivity ($\rho$) and the reactor period ($T$) is fundamental in control rod calibration. The point kinetics equations describe this relationship:
\begin{equation}
    \frac{dP}{dt} = \frac{\rho - \beta}{\Lambda}P + \sum_{i=1}^6 \lambda_i C_i
\end{equation}
\begin{equation}
    \frac{dC_i}{dt} = \frac{\beta_i}{\Lambda}P - \lambda_i C_i
\end{equation}
where:
\begin{itemize}
    \item $P$ is the reactor power,
    \item $\rho$ is the reactivity,
    \item $\beta$ is the delayed neutron fraction,
    \item $\Lambda$ is the prompt neutron lifetime,
    \item $\beta_i$ and $\lambda_i$ are the delayed neutron fractions and decay constants for each of the six precursor groups,
    \item $C_i$ is the concentration of the $i$-th delayed neutron precursor group.
\end{itemize}

We can measure the reactor period $T$ by inserting a control rod and observing the time required for the reactor power to increase by a factor $e$. \\
In practice we will use a factor $1.5$ and compute the period as $T = \frac{\Delta t}{ln(1.5)}$. \\
We can then take advantage of the in-hour equation to correlate the reactivity to the reactor period:
\begin{equation}
    \rho = \frac{\Lambda}{\mathbf{T}} + \sum_{i=1}^6 \frac{\beta_i \lambda_i}{1 + \lambda_i \mathbf{T}}
\end{equation}
We are going to use a 6 group apprach, the values of $\beta_i$ and $\lambda_i$ are known from montecarlo simulations and therefore come with their own uncertanty.

\subsection{Pros and Cons of the Critical Method} 
    \textbf{Advantages}:
    \begin{itemize}
        \item Absolute method: results are independent of other factors, including the position of other control rods
        \item High accuracy of results 
    \end{itemize}
    \textbf{Disadvantages}: 
    \begin{itemize}
        \item Time-intensive, as calibration is required for each control rod
        \item Reactor remains in supercritical condition during measurements
    \end{itemize}

\subsection{Experimental Procedure}
Starting with a reactor in cold  and clean conditions to avoid effects due to external poisons or thermal feedback.
\begin{enumerate}
    \item Bring the reactor to a critical state.
    \item Raise the REG rod by some steps
    \item Measure the time required for the power to increase $3W \rightarrow 4.5W$.
    \item Measure the time required for the power to increase $6W \rightarrow 9W$.
    \item Adjust the shim rod to get back to criticality.
    \item Repeat until the REG rod is fully withdrawn.
\end{enumerate}
It is worth noting that the first measurment $3W \rightarrow 4.5W$ will take into also precursors, by the time 
of the second measurment most of them will be decayed so the measurment should be more rapresentative of the reactiviy. \\