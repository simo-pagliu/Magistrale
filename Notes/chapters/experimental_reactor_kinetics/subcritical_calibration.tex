\section{Subcritical Control Rod Calibration}

\subsection{Introduction}
In reactor physics, the subcritical method provides an alternative approach to control rod calibration, particularly useful for cases where safety is a concern. This method allows for calibration while keeping the reactor in a subcritical state, thus minimizing the risk of reaching supercriticality.

\subsection{Subcritical Calibration Method}
The subcritical method is particularly advantageous for TRIGA reactors, as it enables calibration while maintaining a safe margin. However, this method is also relevant to other types of reactors where supercritical calibration could pose challenges or risks.

\subsection{Theory}
\paragraph{Subcritical Multiplication and Reactor Period}
In subcritical conditions, the neutron population depends on both the external source and the reactor’s multiplication factor $K_{eff}$. This dependency is described by the subcritical multiplication factor $M$:
\begin{equation}
    M = \frac{1}{1 - K_{eff}}
\end{equation}
With an external source, the neutron population evolves over generations until it reaches a steady state. This evolution can be described by a geometric series with $K_{eff}$ as the ratio, where the number of generations required to reach steady state is:
\begin{equation}
    N_{\text{steady}} \approx \frac{4}{\ln K}
\end{equation}

\paragraph{Point Kinetics Equations}
To understand the reactor's response over time, we use the point kinetics equations adapted for the subcritical state:
\begin{equation}
    \frac{dn}{dt} = K_{eff} \left( \frac{K_{eff} - 1}{K_{eff}} + \rho - \beta \right) \frac{n}{\Lambda} + \sum \lambda_i C_i + q
\end{equation}
\begin{equation}
    \frac{dC_i}{dt} = K_{eff} \frac{\beta_i}{\Lambda} n - \lambda_i C_i
\end{equation}
where $\rho = \frac{\Lambda}{K}$ and $q$ is the source term. The reactor period $T$ is then given by:
\begin{equation}
    T = \frac{1}{\lambda_1}
\end{equation}

\subsection{Experimental Procedure}
The following steps outline the subcritical calibration procedure:
\begin{enumerate}
    \item Set the reactor to subcritical with all rods inserted.
    \item Measure the neutron rate $\dot{R}$ as background.
    \item Insert the source, typically $Ra\text{-}Be$, and measure $\dot{R}$ again.
    \item Begin extracting the control rods incrementally and monitor $\dot{R}$.
    \item Compare measured $\dot{R}$ to the calibration parameter $\alpha \Phi$, known from prior calibrations.
    \item Fully extract the shim, wait for steady state, and record.
\end{enumerate}

\begin{tcolorbox}[boxstyle2, title=Pros and Cons]
    \textbf{Advantages}:
    \begin{itemize}
        \item Safe and keeps the reactor subcritical.
        \item Allows calibration of the shim, unlike other methods.
    \end{itemize}
    \textbf{Disadvantages}:
    \begin{itemize}
        \item Lower accuracy due to reliance on source and instrumentation.
        \item Frequent recalibration required due to burnup of fission chamber.
    \end{itemize}
\end{tcolorbox}

\subsection{Results and Analysis}
The reactivity $\rho$ introduced by rod movements is estimated by measuring the counting rate $\dot{R}$ relative to the background and source contributions. The relationship between reactivity and the inverse of the reactor period $T$ allows estimation of $\rho$ from the measured $T$ values:
\begin{equation}
    \rho = \frac{\Lambda}{T} + \sum_{i=1}^{6} \frac{\beta_i \lambda_i}{1 + \lambda_i T}
\end{equation}

\subsection{Conclusion}
The subcritical method provides a robust and safer alternative for control rod calibration. This method allows for steady-state conditions to be reached without supercritical excursions, particularly beneficial for reactors like TRIGA. The balance between accuracy and safety makes it a practical approach for specific calibration needs.