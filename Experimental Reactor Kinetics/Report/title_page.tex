\thispagestyle{plain}

% Title with Titles Font (Anonymous Pro)
{\titlesfont\fontsize{18}{22}\textbf{\color{cherenkovblue}{Control Rod Calibration in TRIGA Reactor \\ Using the Subcritical Multiplication Method}}}\\
{\titlesfont\fontsize{11}{13}\color{cherenkovblue} Nuclear Engineering - Politecnico di Milano}\\

\vspace{-8pt}

% Author information with Main Font (Poppins)
{\normalsize\textbf{Pagliuca Simone, Ronchi Riccardo}} \\
{\footnotesize\textit{simone1.pagliuca@mail.polimi.it, riccardo3.ronchi@mail.polimi.it}} \\

\vspace{8pt}

% Course and Academic Year
{\footnotesize\textbf{Course:} Experimental Reactor Kinetics}\\
{\footnotesize\textbf{Academic Year:} 2024/2025}

% Horizontal rule
\vspace{6pt}
\centerline{\rule{0.95\textwidth}{0.5pt}}
\vspace{8pt}

% Abstract with Titles Font for Heading, Main Font for Body
{\titlesfont\fontsize{10}{12}\textbf{\color{cherenkovblue} ABSTRACT}} 

\vspace{4pt}
{\normalsize
The accurate calibration of control rods is essential for the safe and efficient operation of nuclear reactors. 
This study investigates the subcritical multiplication method for control rod calibration in the TRIGA Mark II reactor at LENA, Pavia. 
A Monte Carlo model was developed using the SERPENT code to estimate control rod worth, and experimental validation was performed. 
The subcritical multiplication method, which relies on neutron source-driven subcritical multiplication, provides a safe and effective approach for reactivity assessment. 

Monte Carlo results exhibited a slight overestimation of control rod worth, which could be mitigated through improved modeling techniques. 
Despite minor discrepancies, the experimental findings are consistent with alternative calibration methodologies.
}

\vspace{10pt}

% Key-words box with Titles Font
\begin{tcolorbox}[arc=0pt, boxrule=0pt, colback=cherenkovblue!60, width=\textwidth, colupper=white]
    {\titlesfont\fontsize{10}{12}\textbf{Key-words:}} Control Rod Calibration, Subcritical Multiplication, TRIGA Reactor, Monte Carlo Simulation
\end{tcolorbox}

\vspace{12pt}

% Nomenclature Definitions
\makenomenclature
\renewcommand\nomgroup[1]{%
\item[\bfseries
\ifstrequal{#1}{N}{Neutronics Quantities}{%
\ifstrequal{#1}{D}{Data \& Statistics}{%
}}%
]}

% Neutronics Quantities
\nomenclature[N]{$\alpha \phi_s$}{Calibration factor (s$^{-1}$)}
\nomenclature[N]{$M$}{Subcritical multiplication factor}
\nomenclature[N]{$K$}{Effective neutron multiplication factor}
\nomenclature[N]{$\rho$}{Reactivity}
\nomenclature[N]{$CRW$}{Control rod worth (dollars)}
\nomenclature[N]{$\Lambda$}{Mean neutron generation time (s)}
\nomenclature[N]{$\beta$}{Delayed neutron fraction}
\nomenclature[N]{$\lambda_i$}{Decay constant for delayed neutron precursor group $i$}
\nomenclature[N]{$\beta_i$}{Delayed neutron fraction for precursor group $i$}
\nomenclature[N]{$T$}{Reactor period (s)}
\nomenclature[N]{$l$}{Prompt neutron lifetime (s)}
\nomenclature[N]{$q$}{Neutron source strength}
\nomenclature[N]{$n$}{Neutron population}

% Data & Statistics
\nomenclature[D]{$C$}{Counts per measurement interval}
\nomenclature[D]{$\dot{C}$}{Counting rate (cps)}
\nomenclature[D]{$\sigma_C$}{Uncertainty in counts}
\nomenclature[D]{MC}{Monte Carlo simulation}
\nomenclature[D]{Exp}{Experimental data}

% Two-column layout for the nomenclature
\renewcommand{\nompreamble}{\begin{multicols}{2}}
\renewcommand{\nompostamble}{\end{multicols}}

% Render Nomenclature Without Section Break
\vspace{-5pt} % Adjust spacing to fit table
\printnomenclature
