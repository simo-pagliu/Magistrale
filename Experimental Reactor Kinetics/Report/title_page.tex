\thispagestyle{plain}

% Title with Titles Font (Anonymous Pro)
{\titlesfont\fontsize{18}{28}\textbf{\color{cherenkovblue}{Control Rod Calibration in TRIGA Reactor Using Subcritical Multiplication Method \\ Experimental Validation and Monte Carlo Analysis}}}\\
{\titlesfont\fontsize{10}{12}\color{cherenkovblue} Nuclear Engineering - Politecnico di Milano}\\

\vspace{-10pt}

% Author information with Main Font (Poppins)
{\normalsize\textbf{Pagliuca Simone, Ronchi Riccardo}} \\
{\footnotesize\textit{simone1.pagliuca@mail.polimi.it, riccardo3.ronchi@mail.polimi.it}} \\

\vspace{10pt}

% Course and Academic Year
{\footnotesize\textbf{Course:} Experimental Reactor Kinetics}\\
{\footnotesize\textbf{Academic year:} 2024/2025}

% Horizontal rule
\vspace{8pt}
\centerline{\rule{1.0\textwidth}{0.4pt}}

\vspace{10pt}

% Abstract with Titles Font for Heading, Main Font for Body
{\fontsize{8}{8}\textbf{\color{cherenkovblue} ABSTRACT:}} 
{\normalsize
The accurate calibration of control rods is essential for the safe and effective operation of nuclear reactors. 
This study focuses on the subcritical method for control rod calibration in the TRIGA Mark II reactor at LENA, Pavia. 
A Monte Carlo model was developed using the SERPENT code to estimate the control rod worth, which was then validated against experimental data. 
The subcritical method, which relies on neutron source-driven subcritical multiplication, offers a safe and efficient approach to reactivity assessment. 

Monte Carlo results exhibit a slight overestimation of control rod worth, which could be mitigated through improved modeling approaches. 

Despite minor discrepancies, the experimental findings remain consistent with results obtained through alternative calibration methods.
}

\vspace{10pt}

% Key-words box with Titles Font
\begin{tcolorbox}[arc=0pt, boxrule=0pt, colback=cherenkovblue!60, width=\textwidth, colupper=white]
    {\titlesfont\fontsize{10}{10}\textbf{Key-words:}} Control Rod Calibration, Subcritical Method, TRIGA Reactor, Monte Carlo Simulation
\end{tcolorbox}

\vspace{10pt}

% Nomenclature Definitions
\makenomenclature
\renewcommand\nomgroup[1]{%
\item[\bfseries
\ifstrequal{#1}{A}{A Quantities}{%
\ifstrequal{#1}{B}{B Quantities}{}}%
]}

% Two-column layout for the nomenclature
\renewcommand{\nompreamble}{\begin{multicols}{2}}
\renewcommand{\nompostamble}{\end{multicols}}

% Define Nomenclature Entries
\nomenclature[A, 01]{$x$}{X quantity}
\nomenclature[B]{$y$}{Y quantity}

% Render Nomenclature Without Section Break
\vspace{-5pt} % Adjust spacing to fit table
\printnomenclature
