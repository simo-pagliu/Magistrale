\section{Introduction}

\subsection{Context}
% Paragraph explaining the broader context of the experiment.

\subsection{Objectives}
We have to determine the breakthrough curve of a knwon step injection of a tracer substance in a column of porous medium.
The tracer is a non-reactive solute, and the column is saturated with water.
The experiment will help us understand the transport properties of the porous medium.
% Paragraph listing the goals of the experiment.

\section{Theoretical Background}
% Paragraph or equations related to the theory supporting the experiment.

\subsection{Relevant Equations and Models}
% Describe models, assumptions, and simplifications used.
Trasmittance index:
\begin{equation}
    T = \frac{I}{I_0} = 10^{-e C L}
\end{equation}
where:  
\begin{itemize}
    \item $T$ is the transmittance index
    \item $I$ is the intensity of the light passing through the solution
    \item $I_0$ is the intensity of the light passing through a reference sample (clean water)
    \item $e$ is the absorption coefficient of the solute
    \item $C$ is the concentration of the solute in the solution
    \item $L$ is the length of the optical path through which light passes in the solution
    \item $C$ is the concentration of the solute in the solution
\end{itemize}

We can then determine the absorbance
\begin{equation}
    A = \log_{10}\left(\frac{1}{T}\right) = e C L
\end{equation}

In practice the Reference sample is associated to an absorbance of $A=1$, 
by observing the difference in ight intensity, knowing the optical path, we can determine 
the relative concentration of the flowing solution:
\begin{equation}
    \frac{A}{A_{ref}} = \frac{C \slashed{eL}}{C_{ref} \slashed{eL}} = \frac{C}{C_{ref}}
\end{equation}

\section{Experimental Setup}

\subsection{Materials and Instruments}
% Description of tools, sensors, and materials used.
The experimental setup consists of a small column designed for didactic purposes, 
ensuring the experiment can be completed in a reasonable time. 
The column is 19.5 cm high and 1 cm in diameter, filled with spherical quartz of known porosity. 
Water flows through the system using a XXX pump, with the desired flow rate set directly on the pump. 
However, the actual flow rate is reduced by the resistance of the pipes and the porous medium. 
To determine the real flow rate, a scale is used to measure the water collected in a beaker placed at the spectrophotometer's exit.

After exiting the column, the water flow passes through a spectrophotometer that measures its transmittance. 
This value is then compared to the transmittance of a reference sample of clean water to obtain a relative measurement. 
The flow is regulated by three valves, ensuring a continuous flow of either clean water or water mixed with a tracer,
 with only one substance passing through the column at a time. For this experiment, a non-reactive tracer is used, 
 specifically a solution of NaNO3 in water.

The entire experimental setup and data gathering process is controlled using LabVIEW on a computer. 
The spectrophotometer and the scale are connected through serial interfaces, 
and the system is configured to take readings every second.

\subsection{Procedure}
% Step-by-step explanation of how the experiment was conducted.
The procedure is programmed to run automatically using LabVIEW. 
The experiment begins by flowing water through the system for a specified duration of X minutes. 
Following this initial phase, a tracer solution is injected into the flow for 480 seconds. 
After the tracer injection, the flow is switched back to water, and the cycle is set to repeat. 
However, only one complete cycle needs to be observed for this experiment.

During the experiment, the relative transmittance of the flowing solution is continuously monitored and compared to that of clean water.
Additionally, the absorbance is calculated using the Lambert-Beer law within the software.

\subsection{Preliminary Evaluation}
To initiate the experiment, we needed to establish a flow rate that would allow us to observe the profile within a reasonable time frame. We estimated the breakthrough time of the profile using the pore velocity vpvp​, which can be calculated as:

\begin{equation}
v_p = \frac{Q}{An}
\end{equation}

where:
\begin{itemize}
    \item $v_p$ is the pore velocity in m/s
    \item $Q$ is the unknown flow rate of the water in m3/s
    \item $A$ is the known cross-sectional area of the column in m2
    \item $n$ is the porosity of the medium (dimensionless)
\end{itemize}

The porosity $n$ can be determined using the mass of water in the column:

\begin{equation}
n = \frac{V_w}{V_{\text{total}}} = \frac{m_w/\rho_w}{V_{\text{total}}} = \frac{m_w}{\rho_w V_{\text{total}}}
\end{equation}

We selected a desired pore velocity $v_p$ such that the breakthrough would pass through the column (ignoring the pipes) in 5 minutes:

\begin{equation}
v_p = \frac{L}{t} = \frac{0.195}{300} = 0.00065 , \text{m/s}
\end{equation}

Using this pore velocity, we computed the required flow rate $Q$:

\begin{equation}
Q = v_p An = 0.00065 \cdot \pi \cdot (0.01)^2 \cdot n = 0.00065 \cdot \pi \cdot (0.01)^2 \cdot 0.4 = 8.16 \times 10^{-6} , \text{m}^3/\text{s}
\end{equation}

This is the actual flow rate we aimed to achieve. To account for the resistance of the system, 
we set a higher flow rate on the pump, settling for $2.002 mL/min$.

\section{Results}

\subsection{Data Collection}
% Raw or pre-processed data in tables or plots.
We obtained a tabe with the reading istant by instant.
\subsection{Data Processing}
% Explain how the data was analyzed, equations used, error handling, etc.

\section{Discussion}

\subsection{Interpretation of Results}
% Discuss what the results mean in the context of the objectives.

\subsection{Sources of Error}
% Identify potential uncertainties or experimental limitations.

\section{Conclusion}
% Summarize findings, evaluate objectives, suggest improvements.